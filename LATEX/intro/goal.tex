After conducting an extensive bibliography research, presented in the \ref{CAP:STATEOFTHEART}, we can conclude all of the studies focus on social media texts. Only \cite{escobar2021gender} introduces a formal datasets, including transcripts of a call center in Venezuela. The main goal of \cite{escobar2021gender} is gender prediction through written text. Therefore the main objective of this study is gender, age and nationality prediction using two datasets: 

\textbf{Twitter dataset}: it contains 5000000 of tweets from users of different spanish-speaking countries: 'enumerate countries'

\textbf{Police dataset}: it contains 'number' of police reports from citizens from 'numberofregions' regions of Spain: 'enumerate regions'

Compared to other studies, we analyze more demographic characteristics (e.g nationality), and we introduce a new dataset containing formal texts, such as police reports. 

In order to provide a high accuracy when predicting the different demographic traits, we carry out the feature selection technique that outperforms the rest of experiments completed in the state of the art \cite{radha2022feature}. We then execute our experiments in various classifiers used in the state of the art to determine which one fits best with our data. Moreover, we perform an experiment introducing the transfer learning technique, only present in \cite{escobar2021gender}. 

Thus, the three main objectives of this paper are: firstly, introduce a new demographic characteristic in the study of author profiling, nationality. Secondly, we study the differences in writing styles in formal texts. To conclude, we analyze if there is enough evidence of linguistic differences between groups of different demographic traits (e.g Spanish regions, gender, age), so that a possible bias might be introduced by AI with an imbalanced training dataset.

Intro meter chicha sobre bias en algoritmos y problematica...es muy recienta preiocupa en eouropa en prensa...que se hace para textos???

 captiolo 3 descripcion de estudio diferencias ty descripsion 2 datasets formal e informal
 capitulo 4 resultados de aplicar el 3 a los datos.

1) Rq3: En el dominio de twitter hay diferencias en el autor segun x,y, z
rq4: en el dominio policial? hay diferencias en el autor segun x,y, z ? y en el sujeto?

rq3 y rq4 buscaria diferencias en escritura? 1 enfoque estadistico? temas? topic modeling? POS verbos, sentece length,
buscar en la literatura forensico como determinar que dos tipos de escritura o tipos de texto son estadisticamente diferentes. Segun el estudio de tal se determina que esto es diferente si cumple x...yo hago este estudio y me da si es diferente o no...buscar la misma persona o no? subseccion de que estudios y tecnias hay para determinar diferencias en escritura.

2) Rq1: aporta informacion la caracersita demografica? Basado en que solo se ha estudiado gender y age?

rq2: son diferentes los texto formales e informales respecto a las diferencias lisngusta a la hora de escribir?

Para demostrarlo haces dos casos de estudio....
%%%%%%%%%% ya sabemos si hay diferencia o no...si no hay diferencias....parece indicar que no hay sesgo si los datos provienen de diferentes genero, edad, o demografia... si si hay habria que ver: i) si impacta en la prediccion, pregunta que vamos a predecir??edad?age?demogrfia? option author pro ii) si se puede corregir con balanceo..o x tecnica...

rq5: hay diferencias en algoritmos si homogeneizamos la muestra? o corregimos las diferencias? 